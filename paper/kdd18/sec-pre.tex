\section{Preliminary}
\label{sec-preliminary}

In this section, we introduce the necessary definitions %of road network and trajectory
and formulate the two problems to be investigated.

\subsection{Basic Definitions}
\label{subsec-definition}

\begin{definition}[Road Network]A road network is a directed graph $\mathcal{N}=(V, E)$ where $V$ denotes the set of road intersections or road ends, and $E\subseteq V\times V$ denotes the set of directed road segments.
%Each (directed) road segment $e=(e.s, e.t)\in E$ is associated with a starting point $e.s\in V$ and an end point $e.t\in V$.
Each road segment $e=(e.s, e.t)$ can only be traveled from its starting point $e.s\in V$ to its end point $e.t\in V$.
\end{definition}

\begin{definition}[Route]A route $\mathcal{R}$ of road network $\mathcal{N}$ is a sequence of connected road segments, $\mathcal{R} = (e_1 \rightarrow e_2 \rightarrow \dots \rightarrow e_n)$ such that $e_i.t=e_{i+1}.s$ ($i\in[1, n-1]$).
\end{definition}

\begin{definition}[Trajectory]A trajectory $\mathcal{T}$ is a sequence of timestamped spatial points, $\mathcal{T} = (p_1 \rightarrow p_2 \rightarrow \dots \rightarrow p_n)$ where each trajectory point $p_i$ ($i\in[1,n]$) contains a timestamp $time_i$ and a geographic coordinate specified by the latitude $lat_i$ and the longitude $lon_i$, \ie $p_i=(time_i,lat_i,lon_i)$.
\end{definition}

In the above we give the most basic definition for trajectory. In practice, however, objects that generate trajectories are very likely to be equipped with various sensors. Thus, each point in a trajectory might contain additional information, \eg driving speed and direction.
%besides the timestamp and geographic coordinate.

%The spatial and temporal information are the most fundamental and important parts in trajectory data. In real life, the objects that generate trajectories are also possible to be equipped with various sensors, like taxicabs with speed sensors. In these situations, each trajectory point can also be assigned with other information besides the timestamp and geographic coordinate, \eg taxicab trajectories with driving speed and direction.


Trajectories and the underlying road network are bonded together through map matching~\cite{Newson2009MM,WuMSZZCWKDD16}. Given a road network $\mathcal{N}$ and a trajectory $\mathcal{T}$, map matching algorithms can find the most likely route $\mathcal{R}$ of $\mathcal{N}$ on which the trajectory $\mathcal{T}$ is generated.
%Map matching is one of the most fundamental and important preprocessing steps in various trajectory data mining tasks~\cite{Zheng2015TDM}.
In addition to the recovered routes, map matching also enables to monitor the traffic condition on road network. After mapping trajectory points onto the corresponding road segments, either the sensor-detected speed or the ratio of the geographic distance to the time lag between two consecutive points can be used to estimate the traveling speeds on specific road segments.
We use the concept of speed snapshot to record traffic condition, which is defined as below.

\begin{definition}[Speed Snapshot]A speed snapshot $G$ of road network $\mathcal{N}=(V, E)$ is a weighted directed graph $G=(V,E,w)$ which has the same structure as $\mathcal{N}$ and the weight $w(e)$ ($e\in E$) is the traveling speed on $e$.
\end{definition}

We construct two types of speed snapshots, \ie {\em real-time speed snapshots} and {\em historical speed snapshots}. %to record real-time and historical traffic conditions, respectively.
%
Since traffic condition varies with respect to time, we first divide the period of time that we consider into different fixed-length time intervals which are consecutively labeled integers from 1 to $T$. Suppose we are considering a period of 30 days and the length of time intervals equals to 10 minutes. We thus have $T=30\times144=4,320$ time intervals.
%
Given a time interval $t\in[1,T]$, \ie time intervals are represented by their labels, a real-time speed snapshot $G_t=(V,E,w_t)$ is constructed to record the real-time traffic condition of time interval $t$. The weight $w_t(e)$ ($e\in E$) averages all the recorded traveling speeds on road segment $e$ within time interval $t$.
%
Similarly, a historical speed snapshot $G'_t=(V,E,w'_t)$ is constructed to record the historical traffic condition corresponding to time interval $t$. We distinguish the historical traffic conditions on weekdays and weekends. That is, the weight $w'_t(e)$ ($e\in E$) averages all the recorded traveling speeds on road segment $e$ during the same time as $t$ on weekdays or holidays, \eg during 8:00 a.m. and 8:10 a.m. on all weekdays. 

%%%%%%%%%%%%%%%%%%% Notation Table
\begin{table}[tb!]
%\vspace{-2ex}
\label{tab-data-stat}
\caption{Summary of Notations}
\vspace{-2.5ex}
\begin{center}
%\begin{small}
\begin{tabular}{|c|l|} \hline
\bf{Notation}   & \bf{Description}   \\ \hline\hline
$\mathcal{N}=(V,E)$ & road network with road intersection set $V$ and road segment set $E$ \\
$\mathcal{R}$ & route \\
$\mathcal{T} = (p_1 \rightarrow p_2 \rightarrow \dots \rightarrow p_n)$ & trajectory with trajectory points $p_i=(time_i,lat_i,lon_i)$\\
$t$, $T$ & time interval, total number of time intervals \\
$G_t$, $G'_t$ & speed snapshots recording real-time and historical traffic condition  \\
\hline
\end{tabular}
\vspace{0ex}
%\end{small}
\end{center}
\end{table}
%%%%%%%%%%%%%%%%%%% Notation Table

For the ease of presentation, Table~1 lists the notations that will be used throughout the paper.

%\stitle{Remarks}.
\subsection{Problem Formulation}
\label{subsec-problem}

We next formally define the two problems studied in this work, namely proactive traffic speed prediction (\ptsp) and collective route planning (\crp).

\begin{definition}[Proactive Traffic Speed Prediction]Given a road network $\mathcal{N}=(V,E)$ and a continuous sequence $(G_1, G_2, \dots, G_T)$ of speed snapshots recording the real-time traffic condition on $\mathcal{N}$, the problem of proactive traffic speed prediction (\ptsp) aims to predict the future traffic speed $w_t(e)$ of arbitrarily road segment in the following $H$ time intervals, \ie $e\in E$ and $t\in[T+1,T+H]$.
\end{definition}

Our \ptsp problem differs from the traditional traffic speed prediction problems in the sense that it
xxx, while traditional traffic speed prediction focuses on predicting the speed at the next time slot or in the near future. {\bf detailed analysis of the difference.}

fixed time points vs. time points in need

Note that the real-time traffic condition is usually collected by static traffic sensors or floating vehicles. Static traffic sensors can only cover a small portion of road segments due to the maintenance cost. On the other hand, more than 80\% of the traffic in a typical city runs on only 10\% to 20\% of the roads~\cite{WuMSZZCWKDD16}. As a result, the incomplete and sparse data becomes a challenge faced by all traffic prediction solutions~\cite{Zhu2013TMC,Shang2014KDD,Lu2017Speed}.


\begin{definition}[Collective Route Planning] Given a sequence of route planning queries, the collective route planning (\crp) problem is to recommend a route to each query such that each recommended route is expected fastest under both the real-time traffic condition and routes for earlier queries.
\end{definition}
