%\stitle{Related work}.
\section{Related work}
\label{sec-related}

This section summarizes related work from three aspects: traffic speed prediction, route planning and others.

\stitle{Traffic speed prediction}. The problem of traffic speed prediction is of great interest in recent years due to both the availability of big traffic data and its fundamental importance in a broad range of urban computing tasks~\cite{Zheng2014TIST}.
%
Since traffic speed is essentially a type of time series data, time series analysis~\cite{time-series-book} has been extensively exploited in this domain, \eg~\cite{traffic2011TRP,TS2012ICDM,HMM2013VLDB,MTL2017ICDM}. In~\cite{TS2012ICDM}, the authors enhance the tradition auto-regression integrated moving average (ARIMA) model to further incorporate the knowledge from both historical data and traffic incidents.

Research efforts have been made to exploit spatiotemporal similarities and road correlations for traffic speed prediction~\cite{Zhu2013TMC,Shang2014KDD,Wang2014KDD,LSM2016KDD,MTL2017ICDM,HMM2013VLDB}.
%
One study~\cite{HMM2013VLDB} uses spatio-temporal hidden Markov models to model correlations among different traffic time series.
%
The studies in~\cite{Zhu2013TMC,Shang2014KDD,Wang2014KDD} first construct a traffic condition matrix or tensor, and then recovers the missing values, \ie traffic speed, through compressive sensing~\cite{Zhu2013TMC}, and matrix/tensor decomposition~\cite{Shang2014KDD,Wang2014KDD}.
%
In~\cite{LSM2016KDD}, the authors explore latent space modeling for road networks and take  into account the network topology as well as temporal and transition patterns.
%
A situation-aware muti-task learning model is proposed in~\cite{MTL2017ICDM}, which clusters speed sensors to automatically identify traffic situations and trains sensors in the same situations as tasks.
%
Recently, deep learning techniques have also been applied, \eg recurrent convolutional neural network~\cite{DL2016ICDM} and long short term memory~\cite{DL2017SDM}.

Our proactive traffic speed prediction model is different from existing studies in two folds. 

%\cite{traffic2011TRP}: multivariate spatial-temporal autoregressive; at a fine granularity (5 min) and over multiple time periods (12 * 5min)
%\cite{TS2012ICDM}: short-term t+1 (1 time interval, e.g., 5min), long-term t+(>1); select either ARIMA or HAM w.r.t. t; retrieval the most similar incident;
%\cite{Zhu2013TMC,Shang2014KDD,Wang2014KDD} compressive sensing, matrix decomposition and sensor decomposition; at current time;
%\cite{LSM2016KDD} Latent Space Model; topology/temporal, fast evolving, dynamic and on-line; both missing at existing and future;
%\cite{MTL2017ICDM} Muti-task Learning each task corresponds to a situation; situation-aware through clustering;
%~\cite{HMM2013VLDB} sparse, dependent and heterogeneous time series; spatial-temporal HMM

\stitle{Route planning}.

\stitle{Others}.




%\stitle{Mobility pattern}.

%\stitle{Traffic Condition Modeling}. On individual road segments. On the entire road network: interpolation-based methods consider spatial correlation among different locations' traffic condition and the compressive sensing-based approaches~\cite{Zhu2013TMC} consider both spatial and temporal correlation (by filling empty values in a road-time matrix). The work in~\cite{Shang2014KDD} further incorporates two sets of additional knowledge, \ie geographic contexts of road segments and traffic pattern revealing the correlations between different time slots.

%\stitle{Traffic Congestion Propagation}. \cite{Nguyen2017TBD} discovers frequent patterns of congestion propagations in the traffic networks by a) discovering spatio-temporal congestions and causal relationships between them, b) revealing recurrent congestion patterns in the road network based on frequent subtree algorithm, and c) modelling the traffic congestion propagation probability. [SIGSPATIAL'17] takes a road network, the congestion conditions (when becoming congested, 20km/h) on each road segment spanning several days as input, and aims to infer the cascading patterns between road segments. 