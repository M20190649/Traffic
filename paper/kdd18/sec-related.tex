%\stitle{Related work}. 
\section{Related work}
\label{sec-related}
We summarize related work as follows.

\stitle{Mobility pattern}.

\stitle{Traffic Condition Modeling}. On individual road segments. On the entire road network: interpolation-based methods consider spatial correlation among different locations' traffic condition and the compressive sensing-based approaches~\cite{Zhu2013TMC} consider both spatial and temporal correlation (by filling empty values in a road-time matrix). The work in~\cite{Shang2014KDD} further incorporates two sets of additional knowledge, \ie geographic contexts of road segments and traffic pattern revealing the correlations between different time slots.

\stitle{Traffic Congestion Propagation}. \cite{Nguyen2017TBD} discovers frequent patterns of congestion propagations in the traffic networks by a) discovering spatio-temporal congestions and causal relationships between them, b) revealing recurrent congestion patterns in the road network based on frequent subtree algorithm, and c) modelling the traffic congestion propagation probability. [SIGSPATIAL'17] takes a road network, the congestion conditions (when becoming congested, 20km/h) on each road segment spanning several days as input, and aims to infer the cascading patterns between road segments.
