%\stitle{Related work}.
\section{Related work}
\label{sec-related}

We summarize related work as follows.

\stitle{Traffic Speed Prediction}. The problem of traffic prediction is of great interest in recent years due to both the availability of big traffic data and its fundamental importance for a broad range of urban computing tasks~\cite{Zheng2014TIST}.
%
Since traffic speed is essentially a type of time series data, time series analysis has been extensively exploited in this domain~\cite{traffic2011TRP,TS2012ICDM}. Notably, in~\cite{TS2012ICDM} the authors enhance the tradition auto-regression integrated moving average (ARIMA) model to further incorporate the knowledge from both historical data and traffic incidents.


Latent Space Model~\cite{LSM2016KDD}.
Muti-task Learning~\cite{MTL2017ICDM}.
Matrix decomposition~\cite{Zhu2013TMC,Shang2014KDD} and Tensor decomposition~\cite{Wang2014KDD}.
Deep Learning~\cite{DL2016ICDM,DL2017SDM,DL2017CoRR}

%\cite{traffic2011TRP}: multivariate spatial-temporal autoregressive, at a fine granularity (5 min) and over multiple time periods (12 * 5min)
%\cite{TS2012ICDM}: short-term t+1 (1 time interval, e.g., 5min), long-term t+(>1)

\stitle{Route Planning}.

\stitle{Others}. 




%\stitle{Mobility pattern}.

%\stitle{Traffic Condition Modeling}. On individual road segments. On the entire road network: interpolation-based methods consider spatial correlation among different locations' traffic condition and the compressive sensing-based approaches~\cite{Zhu2013TMC} consider both spatial and temporal correlation (by filling empty values in a road-time matrix). The work in~\cite{Shang2014KDD} further incorporates two sets of additional knowledge, \ie geographic contexts of road segments and traffic pattern revealing the correlations between different time slots.

%\stitle{Traffic Congestion Propagation}. \cite{Nguyen2017TBD} discovers frequent patterns of congestion propagations in the traffic networks by a) discovering spatio-temporal congestions and causal relationships between them, b) revealing recurrent congestion patterns in the road network based on frequent subtree algorithm, and c) modelling the traffic congestion propagation probability. [SIGSPATIAL'17] takes a road network, the congestion conditions (when becoming congested, 20km/h) on each road segment spanning several days as input, and aims to infer the cascading patterns between road segments. 