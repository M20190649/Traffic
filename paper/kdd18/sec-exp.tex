\section{Experimental Study}
\label{sec-exp}

Using two real-life datasets, we conduct an extensive experimental study of , compared with the state of the art methods .


\subsection{Experimental Settings}

We first introduce the settings of our experimental study.

\stitle{Datasets}. We chose two datasets, \ie~\beijing and \shenzhen,  to test our approach.
%
Each of the two datasets consists of the road network\footnote{https://mapzen.com/data/metro-extracts/metro/beijing\_china/ \& https://mapzen.com/data/metro-extracts/metro/shenzhen\_china/} of the corresponding city in China (\ie city Beijing and city Shenzhen) and the taxi trajectory data. The trajectories of both \beijing and \shenzhen are collected in three months. Table~1 gives a summary of some statistics on the two datasets.

%\beijing Range: minlat="39.414" minlon="115.686" maxlat="40.426" maxlon="117.119", i.e., 112.5km(1.012) x 123.0(1.433)km
%road network nodes: 96406
%road network directed edges: 216153
%road network undirected edges: 129955
%total distance (km): 31528.9246597
%weakly connected component: 1465 (91223, 58, 45, ...)

%%%%%%%%%%%%%%%%%%% Dataset Statistics Table
\begin{table}[tb!]
%\vspace{-2ex}
\label{tab-data-stat}
\caption{Summary of Dataset Statistics}
\vspace{-2.5ex}
\begin{center}
%\begin{small}
\begin{tabular}{|c|c|c|c|c|c|} \hline
&  \multicolumn{2}{c|}{\bf Road Network}   & \multicolumn{3}{c|}{\bf Trajectory Data}   \\ \cline{2-6}
\raisebox{1ex}[0pt]{\bf Datasets} & {\bf \# of Nodes} & {\bf \# of Edges} & {\bf \# of Taxicabs} & {\bf Time Span} & {\bf \# of Points} \\
\hline
\beijing      &  96,406 & 216,153 & 12,791 & 12/01/2013 -- 02/28/2014 & x          \\ \hline
\shenzhen      &  80,771 & 176,112 & 2,544 & 04/01/2011 -- 06/30/2011 & y       \\ \hline
\end{tabular}
\vspace{0ex}
%\end{small}
\end{center}
\footnotesize Note: we presents the number of {\em directed} edges in the road network.
\end{table}
%%%%%%%%%%%%%%%%%%% Dataset Statistics Table


Each trajectory point contains: (a) a timestamp, (b) a location specified by the latitude and the longitude and (c) a driving status describing the speed and the direction. We preprocessed the two datasets by filtering out stay points and out-of-bounds points from the trajectory data and using the map matching algorithm~\cite{Newson2009MM} to derive a sequence of traveled road segments given a trajectory. The speed of each trajectory point was then assigned to the closest road segment in the matched sequence.


\stitle{Algorithms and implementation}.
Algorithms were all implemented with Java, including the state of the art algorithm  and the synthetic data generator that are available at {\small http://www.cs.ucsb.edu/\~dbl/software.php}.

All experiments were run on a PC with 2 Intel Xeon
E5--2630 2.4GHz CPUs and 64 GB of memory. The usage of virtual memory was forbidden in all our tests. When quantity measures are evaluated, the test was repeated over 5 times and the average is reported here.



\subsection{Experimental Results}

We next present our findings.
