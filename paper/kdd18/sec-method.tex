\section{Methodology}
\label{sec-method}


\subsection{Proactive Traffic Prediction}
\label{subsec-proactive}

Given the speed of road segments (represented by a speed vector $v$) at the current time, predict the speed in the near future: $v'=Mv+b$.

Key: local transmission, \ie sparsity / time dependent / fully data-driven (i. number of time slots and ii. probability by frequency and beyond)


\stitle{Learning speed transmission}. We present two methods to learn the local transmission.

\etitle{Trajectory}.
The first method learns the local transmission from trajectories. The intuition behind is that 

More specifically, given a trajectory $Tr$ generated by a taxicab, the corresponding sequence of road segments is first revealed through map matching methods~\cite{Newson2009MM}. Nearby road segments are then captured by a sliding window. Considering that the length of road segments varies significantly, we further let the length of the sliding window, \ie the number of road segments in the windows, be adaptive to the road segment length. 

\begin{example}
learn matrix M from trajectories. 
\end{example}

\etitle{Topology}. 
The second method learns the local transmission from the topology of the underlying road network, under the assumption that drivers always drive along the shortest paths from their origins to their destinations. This assumption is widely used in tasks of urban computing, \eg map matching. 

{\bf Method.} 1. Randomly sample a starting point; 2. BFS shortest paths to other nodes; 3. use the shortest paths along the same line as sequence of road segments corresponding to trajectories.






\subsection{Collective Route Planning}
\label{subsec-route}

