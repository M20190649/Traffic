\documentclass[acmlarge]{acmart}

\usepackage{booktabs} % For formal tables

%\special{papersize=8.5in,11in}
%\usepackage{latexsym}
%\usepackage{amsfonts}
%\usepackage{amsmath}
%\usepackage{amssymb}
\usepackage{color}
\usepackage{epsfig}
\usepackage{xspace}
\usepackage{graphicx,epstopdf}
\usepackage{subfigure}
%\usepackage{cite}
\usepackage{balance}
%\usepackage{amsmath, bm}
\usepackage[english]{babel}

\pdfpagewidth=8.5in
\pdfpageheight=11in

%%For removing copyright
%\usepackage{blindtext}
%\usepackage{etoolbox}
%\makeatletter
%\patchcmd{\maketitle}{\@copyrightspace}{}{}{}
%\makeatother
%%%%%End of removing copyright


% ALGORITHMS
%%%%%%%%%%%%%%%%%%%%%%%%%%%%%%%%%%%%%%%%%%%%%%%%%%%%%%%%%%%%%%%%%%%%%%%%%%%%%%%
\newcommand{\SELECT}{\mbox{{\bf select}}\ }
\newcommand{\FROM}{\mbox{{\bf from}\ }}
\newcommand{\WHERE}{\mbox{\bf where}\ }
\newcommand{\SUM}{\mbox{{\bf sum}}\ }
\newcommand{\GROUPBY}{\mbox{{\bf group by}}\ }
\newcommand{\HAVING}{\mbox{{\bf having}}\ }
\newcommand{\CASE}{\mbox{{\bf case}}\ }
\newcommand{\END}{\mbox{{\bf end}}\ }
\newcommand{\WHEN}{\mbox{{\bf when}}\ }
\newcommand{\EXISTS}{\mbox{{\bf exists}}\ }
\newcommand{\COUNT}{\mbox{\kw{count}}}
\newcommand{\INSERTINTO}{\mbox{{\bf insert into}}\ }
\newcommand{\UPDATE}{\mbox{{\bf update}}\ }
\newcommand{\SET}{\mbox{{\bf set}}\ }
\newcommand{\IN}{\mbox{{\bf in}}\ }
\newcommand{\If}{\mbox{\bf if}\ }
\newcommand{\Let}{\mbox{\bf let}\ }
\newcommand{\Call}{\mbox{\bf call}\ }
\newcommand{\Then}{\mbox{\bf then}\ }
\newcommand{\To}{\mbox{\bf to}\ }
\newcommand{\Else}{\mbox{\bf else}\ }
\newcommand{\ElseIf}{\mbox{\bf elseif}\ }

\newcommand{\While}{\mbox{\bf while}\ }
\newcommand{\Begin}{\mbox{\bf begin}\ }
\newcommand{\End}{\mbox{\bf end}\ }
\newcommand{\Do}{\mbox{\bf do}\ }
\newcommand{\Downto}{\mbox{\bf downto}\ }
\newcommand{\Repeat}{\mbox{\bf repeat}\ }
\newcommand{\Until}{\mbox{\bf until}\ }
\newcommand{\For}{\mbox{\bf for}\ }
\newcommand{\Merge}{\mbox{\bf merge}\ }
\newcommand{\Replace}{\mbox{\bf replace}\ }
\newcommand{\Remove}{\mbox{\bf remove}\ }
\newcommand{\Find}{\mbox{\bf find}\ }
\newcommand{\Each}{\mbox{\bf each}\ }

\newcommand{\ForEach}{\mbox{\bf for each}\ }
\newcommand{\Or}{\mbox{\bf or}\ }
\renewcommand{\And}{\mbox{\bf and}\ }
\newcommand{\Not}{\mbox{\bf not}\ }
\newcommand{\Break}{\mbox{\bf break}\ }
\newcommand{\Continue}{\mbox{\bf continue}\xspace}
\newcommand{\Return}{\mbox{\bf return}\ }
\newcommand{\Case}{\mbox{\bf case}\ }
\newcommand{\Of}{\mbox{\bf of}\ }
\newcommand{\EndCase}{\mbox{\bf end-case}\ }
\newcommand{\NIL}{\mbox{\em nil}}
\newcommand{\False}{\mbox{\em false}}
\newcommand{\True}{\mbox{\em true}}
\newcommand{\algAND}{{\sc and}\xspace}
\newcommand{\OR}{{\sc or}\xspace}
\newcommand{\NOT}{{\sc not}\xspace}
%%%%%%%%%%%%%%%%%%%%%%%%%%%%%%%%%%%%%%%%%%%%%%%%%%%%%%%%%%%%%%%%% ALGORITHMS

\newcommand{\eat}[1]{}
\newcommand{\sttab}{\rule{0pt}{8pt}\\[-3ex]}
\newcounter{ccc}
\newcommand{\bcc}{\setcounter{ccc}{1}\theccc.}
\newcommand{\icc}{\addtocounter{ccc}{1}\theccc.}
\newcommand{\myhrule}{\rule[.5pt]{\hsize}{.5pt}}
\newcommand{\mat}[2]{{\begin{tabbing}\hspace{#1}\=\+\kill #2\end{tabbing}}}
\newcommand{\stitle}[1]{\vspace{0.5ex}\noindent{\bf #1}}
\newcommand{\etitle}[1]{\vspace{0.5ex}\noindent{\em \underline{#1}}}
\newcommand{\marked}[1]{\textcolor{red}{#1}}
\newcommand{\markedb}[1]{\textcolor{blue}{#1}}
\newcommand{\subsubtitle}[1]{\vspace{0.5ex}\noindent\underline{{\bf #1}}}

%\newcommand{\stab}{\rule{0pt}{8pt}\\[-1.6ex]}
%\newcommand{\sttab}{\rule{0pt}{8pt}\\[-2ex]}
\newcommand{\sstab}{\rule{0pt}{8pt}\\[-2ex]}
\newcommand{\bi}{\begin{itemize}}
\newcommand{\ei}{\end{itemize}}
\renewcommand{\ni}{\noindent}


\newcommand{\ie}{\emph{i.e.,}\xspace}
\newcommand{\eg}{\emph{e.g.,}\xspace}
\newcommand{\wrt}{\emph{w.r.t.}\xspace}
\newcommand{\aka}{\emph{a.k.a.}\xspace}
\newcommand{\kw}[1]{{\ensuremath {\mathsf{#1}}}\xspace}

%problems
\newcommand{\ptsp}{{\sc ptsp}\xspace}
\newcommand{\crp}{{\sc crp}\xspace}

%concepts
\newcommand{\scc}{{\sc SCC}\xspace}
\newcommand{\snapshot}[1]{\widehat{#1}}

%data set
\newcommand{\beijing}{{\sc Beijing}\xspace}
\newcommand{\shenzhen}{{\sc Shenzhen}\xspace}
\newcommand{\sanfan}{{\sc San Francisco}\xspace}

%baselines
\newcommand{\pagerank}{\kw{PRank}\xspace} %PageRank

%metrics
\newcommand{\PairAcc}{\kw{PairAcc}\xspace}




\usepackage[ruled]{algorithm2e} % For algorithms
\renewcommand{\algorithmcfname}{ALGORITHM}
\SetAlFnt{\small}
\SetAlCapFnt{\small}
\SetAlCapNameFnt{\small}
\SetAlCapHSkip{0pt}
\IncMargin{-\parindent}

% Metadata Information
\acmJournal{IMWUT}
\acmVolume{0}
\acmNumber{0}
\acmArticle{0}
\acmYear{2018}
\acmMonth{3}
\acmArticleSeq{00}

%\acmBadgeR[http://ctuning.org/ae/ppopp2016.html]{ae-logo}
%\acmBadgeL[http://ctuning.org/ae/ppopp2016.html]{ae-logo}


% Copyright
%\setcopyright{acmcopyright}
%\setcopyright{acmlicensed}
%\setcopyright{rightsretained}
%\setcopyright{usgov}
\setcopyright{usgovmixed}
%\setcopyright{cagov}
%\setcopyright{cagovmixed}

% DOI
\acmDOI{0000001.0000001}

% Paper history
\received{February 2018}
\received{XXXX 2018}
\received[accepted]{XXXX 2018}


% Document starts
\begin{document}
% Title portion
\title{Multi-Scenario Traffic Speed and Flow Prediction}

\title{Towards Expected Fastest Route Recommendation}

\title{A Flow-based Speed Prediction Approach to Recommending Expected Fastest Routes}

\title{Collective Route Planning through Proactive Traffic Speed Prediction}
%\titlenote{We can add a note to the title}


\eat{ %%%%Double Blind
\author{Renjun Hu}
\orcid{1234-5678-9012-3456}
\affiliation{%
  \institution{SKLSDE Lab, Beihang University}
  \streetaddress{37 Xueyuan Rd}
  \city{Haidian}
  \state{Beijing}
  \country{China}
  \postcode{100191}}
\email{hurenjun@buaa.edu.cn}
\author{Junming Liu}
\affiliation{%
  \institution{Rutgers University} %Rutgers Business School,
  %\department{}
  \streetaddress{1 Washington Park}
  \city{Newark}
  \state{NJ}
  \postcode{07102}
  \country{USA}}
\email{jl1433@rutgers.edu}
\author{Yanchi Liu}
\affiliation{%
  \institution{Rutgers University} %Rutgers Business School,
  %\department{}
  \streetaddress{1 Washington Park}
  \city{Newark}
  \state{NJ}
  \postcode{07102}
  \country{USA}}
\email{yanchi.liu@rutgers.edu}
\author{Hui Xiong}
\affiliation{%
  \institution{Rutgers University} %Rutgers Business School,
  %\department{}
  \streetaddress{1 Washington Park}
  \city{Newark}
  \state{NJ}
  \postcode{07102}
  \country{USA}}
\email{hxiong@rutgers.edu}
\author{Shuai Ma}
%\authornote{This is the corresponding author}
\affiliation{%
  \institution{SKLSDE Lab, Beihang University}
  \streetaddress{37 Xueyuan Rd}
  \city{Haidian}
  \state{Beijing}
  \country{China}
  \postcode{100191}}
\email{mashuai@buaa.edu.cn}
\author{Jinpeng Huai}
\affiliation{%
  \institution{SKLSDE Lab, Beihang University}
  \streetaddress{37 Xueyuan Rd}
  \city{Haidian}
  \state{Beijing}
  \country{China}
  \postcode{100191}}
\email{huaijp@buaa.edu.cn}
}

\author{Submitted for blind review}





\eat{   %%% conference short author list
\author{
Renjun Hu$^{1,2}$, Junming Liu$^3$, Hui Xiong$^3$, Shuai Ma$^{1,2}$, Jinpeng Huai$^{1,2}$}
\affiliation{%
  \institution{$^1$ SKLSDE Lab, Beihang University, China}
  \institution{$^2$ Beijing Advanced Innovation Center for Big Data and Brain Computing, China}
  \institution{$^3$ Rutgers University, USA}
  \institution{\{hurenjun, mashuai, huaijp\}@buaa.edu.cn \hspace{12ex} \{jl1433, hxiong\}@rutgers.edu}
}
}%%%%%EAT



\begin{abstract}
Abstract.
\end{abstract}


%
% The code below should be generated by the tool at
% http://dl.acm.org/ccs.cfm
% Please copy and paste the code instead of the example below.
%
\eat{
\begin{CCSXML}
<ccs2012>
 <concept>
  <concept_id>10010520.10010553.10010562</concept_id>
  <concept_desc>Computer systems organization~Embedded systems</concept_desc>
  <concept_significance>500</concept_significance>
 </concept>
 <concept>
  <concept_id>10010520.10010575.10010755</concept_id>
  <concept_desc>Computer systems organization~Redundancy</concept_desc>
  <concept_significance>300</concept_significance>
 </concept>
 <concept>
  <concept_id>10010520.10010553.10010554</concept_id>
  <concept_desc>Computer systems organization~Robotics</concept_desc>
  <concept_significance>100</concept_significance>
 </concept>
 <concept>
  <concept_id>10003033.10003083.10003095</concept_id>
  <concept_desc>Networks~Network reliability</concept_desc>
  <concept_significance>100</concept_significance>
 </concept>
</ccs2012>
\end{CCSXML}

\ccsdesc[500]{Computer systems organization~Embedded systems}
\ccsdesc[300]{Computer systems organization~Redundancy}
\ccsdesc{Computer systems organization~Robotics}
\ccsdesc[100]{Networks~Network reliability}

%
% End generated code
%
}%%%%EAT

%\keywords{Wireless sensor networks, media access control,
%multi-channel, radio interference, time synchronization}

% DO NOT use this command unless you want to change
% the default behavior
% \authorsaddresses{Authors' addresses: G.~Zhou, Computer Science
%   Department, College of William and Mary, 104 Jameson Rd,
%   Williamsburg, PA 23185, US, \path{gzhou@wm.edu}; V.~B\'eranger,
%   Inria Paris-Rocquencourt, Rocquencourt, France; A.~Patel, Rajiv
%   Gandhi University, Rono-Hills, Doimukh, Arunachal Pradesh, India;
%   H.~Chan, Tsinghua University, 30 Shuangqing Rd, Haidian Qu, Beijing
%   Shi, China; T.~Yan, Eaton Innovation Center, Prague, Czech Republic;
%   T.~He, C.~Huang, J.~A.~Stankovic University of Virginia, School of
%   Engineering Charlottesville, VA 22903, USA; T. F. Abdelzaher,
%   (Current address) NASA Ames Research Center, Moffett Field,
%   California 94035.}

\maketitle

% The default list of authors is too long for headers.
\renewcommand{\shortauthors}{R. Hu et al.}


\section{Introduction}
\label{sec-intro}

Modeling traffic flows is one of the most fundamental issues for urban transportation problems. It aims to continuously predict the traffic flows on each road segments. An accurate traffic flow prediction model can benefit numerous real-life tasks, such as travel time estimation, intelligent driving direction, traffic congestion detection and urban planning~\cite{Zheng2014TIST}.

Traffic flow is XXXXXXX in real-life. {\em reasons and examples of dynamism.} {\em problems of ignoring the dynamism}

{\bf Existing models} predict the flows on road segments at specific time slot. The main idea behind these models is that ``similar" road segments share similar traffic condition, and road segments at the same time slot share similar traffic. Specifically, these model exploit low rank matrix decomposition to reveal the hidden structure of traffic condition matrix which records the traffic condition of each road segment spanning a period of time. However, the derived latent factors only preserve the main structure of the matrix, and, hence, cannot deal with the traffic flows in {\em anomalous cases}, \eg road works, events, accidents.

\stitle{Goal}. To predict the city-wide traffic condition, \ie the number of vehicles and the traveling speed of each road segments.

Modeling of traffic condition:
a)	the flow speed on this link within the time slot to indicate this traffic condition;
b)	both travel speed and traffic volume;
c)	Density, length of queue, etc.

\stitle{Challenges}.
\bi
\item Traffic is xxxxxxx, \eg traffic accident, road work, administrative control, and event, holiday (volume and capacity)
\item detour behaviors
\item Data problems: Data is sparse and uneven on both time and space (topology + trajectory), Individual trajectory is noisy and unreliable (group behavior)
\ei

\stitle{Contributions}. To this end,

\ni (1) Data-driven transmission model. For each intersection, total in-flow and out-flows, and further distinguish the normal and anomalous flows, spatio-temporal correlations

\ni (2) prediction model

\ni (3)


\section{Preliminary}
\label{sec-preliminary}

In this section, we introduce the necessary definitions %of road network and trajectory
and formulate the two problems to be investigated.

\subsection{Basic Definitions}
\label{subsec-definition}

\begin{definition}[Road Network]A road network is a directed graph $\mathcal{N}=(V, E)$ where $V$ denotes the set of road intersections or road ends, and $E\subseteq V\times V$ denotes the set of directed road segments.
%Each (directed) road segment $e=(e.s, e.t)\in E$ is associated with a starting point $e.s\in V$ and an end point $e.t\in V$.
Each road segment $e=(e.s, e.t)$ can only be traveled from its starting point $e.s\in V$ to its end point $e.t\in V$.
\end{definition}

\begin{definition}[Route]A route $\mathcal{R}$ of road network $\mathcal{N}$ is a sequence of connected road segments, $\mathcal{R} = (e_1 \rightarrow e_2 \rightarrow \dots \rightarrow e_n)$ such that $e_i.t=e_{i+1}.s$ ($i\in[1, n-1]$).
\end{definition}

\begin{definition}[Trajectory]A trajectory $\mathcal{T}$ is a sequence of timestamped spatial points, $\mathcal{T} = (p_1 \rightarrow p_2 \rightarrow \dots \rightarrow p_n)$ where each trajectory point $p_i$ ($i\in[1,n]$) contains a timestamp $time_i$ and a geographic coordinate specified by the latitude $lat_i$ and the longitude $lon_i$, \ie $p_i=(time_i,lat_i,lon_i)$.
\end{definition}

In the above we give the most basic definition for trajectory. In practice, however, objects that generate trajectories are very likely to be equipped with various sensors. Thus, each point in a trajectory might contain additional information, \eg driving speed and direction.
%besides the timestamp and geographic coordinate.

%The spatial and temporal information are the most fundamental and important parts in trajectory data. In real life, the objects that generate trajectories are also possible to be equipped with various sensors, like taxicabs with speed sensors. In these situations, each trajectory point can also be assigned with other information besides the timestamp and geographic coordinate, \eg taxicab trajectories with driving speed and direction.


Trajectories and the underlying road network are bonded together through map matching~\cite{Newson2009MM,WuMSZZCWKDD16}. Given a road network $\mathcal{N}$ and a trajectory $\mathcal{T}$, map matching algorithms can find the most likely route $\mathcal{R}$ of $\mathcal{N}$ on which the trajectory $\mathcal{T}$ is generated.
%Map matching is one of the most fundamental and important preprocessing steps in various trajectory data mining tasks~\cite{Zheng2015TDM}.
In addition to the recovered routes, map matching also enables to monitor the traffic condition on road network. After mapping trajectory points onto the corresponding road segments, either (i) the sensor-detected speed or (ii) the ratio of the geographic distance to the time lag between two consecutive points can be used to estimate the traveling speeds on specific road segments.
We use the concept of speed snapshot to record traffic condition, which is defined as below.

\begin{definition}[Speed Snapshot]A speed snapshot $G$ of road network $\mathcal{N}=(V, E)$ is a weighted directed graph $G=(V,E,w)$ which has the same structure as $\mathcal{N}$ and the weight $w(e)$ ($e\in E$) is the traveling speed on $e$.
\end{definition}

Since traffic condition varies with respect to time, we divide the period of time that we consider into different fixed-length time intervals and consecutively label these time intervals from 1 to $T$. Suppose we are considering a period of 30 days and the length of time intervals equals to 10 minutes. We thus have $T=30\times144=4,320$ time intervals.
Given a time interval $t\in[1,T]$, a speed snapshot $G_t=(V,E,w_t)$ is constructed to record the {\em real-time} traffic condition of time interval $t$ such that weight $w_t(e)$ ($e\in E$) averages all the recorded traveling speeds on road segment $e$ during $t$.
Further, let $G'_t=(V,E,w'_t)$ record the {\em historical} traffic condition corresponding to time interval $t$ where weight $w'_t(e)$ ($e\in E$) averages all the recorded traveling speeds on road segment $e$ during the same time as $t$. For instance, let $t$ represent the time interval 8:00 am to 8:10 am

%%%%%%%%%%%%%%%%%%% Notation Table
\begin{table}[tb!]
%\vspace{-2ex}
\label{tab-data-stat}
\caption{Summary of Notations}
\vspace{-2.5ex}
\begin{center}
%\begin{small}
\begin{tabular}{|c|l|} \hline
\bf{Notation}   & \bf{Description}   \\ \hline\hline 
$\mathcal{N}=(V,E)$ & road network with road intersection set $V$ and road segment set $E$ \\ 
$\mathcal{R}$ & route \\
$\mathcal{T} = (p_1 \rightarrow p_2 \rightarrow \dots \rightarrow p_n)$ & trajectory with trajectory points $p_i=(time_i,lat_i,lon_i)$\\
$t$, $T$ & time interval, total number of time intervals \\
$G_t$, $G'_t$ & speed snapshots recording real-time and historical traffic condition  \\
\hline
\end{tabular}
\vspace{0ex}
%\end{small}
\end{center}
\end{table}
%%%%%%%%%%%%%%%%%%% Notation Table

For the ease of presentation, Table~1 lists the notations that will be used throughout the paper.

%\stitle{Remarks}. 
\subsection{Problem Formulation}
\label{subsec-problem}

We next formally define the two problems studied in this work, namely proactive traffic speed prediction (\ptsp) and collective route planning (\crp).

\begin{definition}[Proactive Traffic Speed Prediction]Given a road network $\mathcal{N}=(V,E)$ and a continuous sequence $(G_1, G_2, \dots, G_T)$ of speed snapshots recording the real-time traffic condition on $\mathcal{N}$, the problem of proactive traffic speed prediction (\ptsp) aims to predict the future traffic speed $w_t(e)$ of arbitrarily road segment in the following $H$ time intervals, \ie $e\in E$ and $t\in[T+1,T+H]$.
\end{definition}

Our \ptsp problem differs from the traditional traffic speed prediction problems in the sense that it
xxx, while traditional traffic speed prediction focuses on predicting the speed at the next time slot or in the near future. {\bf detailed analysis of the difference.}

fixed time points vs. time points in need

Note that the real-time traffic condition is usually collected by static traffic sensors or floating vehicles. Static traffic sensors can only cover a small portion of road segments due to the maintenance cost. On the other hand, more than 80\% of the traffic in a typical city runs on only 10\% to 20\% of the roads~\cite{WuMSZZCWKDD16}. As a result, the incomplete and sparse data becomes a challenge faced by all traffic prediction solutions~\cite{Zhu2013TMC,Shang2014KDD,Lu2017Speed}.


\begin{definition}[Collective Route Planning] Given a sequence of route planning queries, the collective route planning (\crp) problem is to recommend a route to each query such that each recommended route is expected fastest under both the real-time traffic condition and routes for earlier queries.
\end{definition}

\section{Methodology}
\label{sec-method}


\subsection{Proactive Traffic Prediction}
\label{subsec-proactive}

Given the speed of road segments (represented by a speed vector $v$) at the current time, predict the speed in the near future: $v'=Mv+b$.

Key: local transmission, \ie sparsity / time dependent / fully data-driven (i. number of time slots and ii. probability by frequency and beyond)


\stitle{Learning speed transmission}. We present two methods to learn the local transmission.

\etitle{Trajectory}.
The first method learns the local transmission from trajectories. The intuition behind is that 

More specifically, given a trajectory $Tr$ generated by a taxicab, the corresponding sequence of road segments is first revealed through map matching methods~\cite{Newson2009MM}. Nearby road segments are then captured by a sliding window. Considering that the length of road segments varies significantly, we further let the length of the sliding window, \ie the number of road segments in the windows, be adaptive to the road segment length. 

\begin{example}
learn matrix M from trajectories. 
\end{example}

\etitle{Topology}. 
The second method learns the local transmission from the topology of the underlying road network, under the assumption that drivers always drive along the shortest paths from their origins to their destinations. This assumption is widely used in tasks of urban computing, \eg map matching. 

{\bf Method.} 1. Randomly sample a starting point; 2. BFS shortest paths to other nodes; 3. use the shortest paths along the same line as sequence of road segments corresponding to trajectories.






\subsection{Collective Route Planning}
\label{subsec-route}


\section{Collective Route Planning}
\label{sec-route}
\section{Experimental Study}
\label{sec-exp}

Using two real-life datasets, we conduct an extensive experimental study of , compared with the state of the art methods.




%\stitle{Related work}.
\section{Related work}
\label{sec-related}

We summarize related work as follows.

\stitle{Traffic Speed Prediction}. The problem of traffic prediction is of great interest in recent years due to both the availability of big traffic data and its fundamental importance for a broad range of urban computing tasks~\cite{Zheng2014TIST}.
%
Since traffic speed is essentially a type of time series data, time series analysis has been extensively exploited in this domain~\cite{traffic2011TRP,TS2012ICDM}. Notably, in~\cite{TS2012ICDM} the authors enhance the tradition auto-regression integrated moving average (ARIMA) model to further incorporate the knowledge from both historical data and traffic incidents.


Latent Space Model~\cite{LSM2016KDD}.
Muti-task Learning~\cite{MTL2017ICDM}.
Matrix decomposition~\cite{Zhu2013TMC,Shang2014KDD} and Tensor decomposition~\cite{Wang2014KDD}.
Deep Learning~\cite{DL2016ICDM,DL2017SDM,DL2017CoRR}

%\cite{traffic2011TRP}: multivariate spatial-temporal autoregressive, at a fine granularity (5 min) and over multiple time periods (12 * 5min)
%\cite{TS2012ICDM}: short-term t+1 (1 time interval, e.g., 5min), long-term t+(>1)

\stitle{Route Planning}.

\stitle{Others}. 




%\stitle{Mobility pattern}.

%\stitle{Traffic Condition Modeling}. On individual road segments. On the entire road network: interpolation-based methods consider spatial correlation among different locations' traffic condition and the compressive sensing-based approaches~\cite{Zhu2013TMC} consider both spatial and temporal correlation (by filling empty values in a road-time matrix). The work in~\cite{Shang2014KDD} further incorporates two sets of additional knowledge, \ie geographic contexts of road segments and traffic pattern revealing the correlations between different time slots.

%\stitle{Traffic Congestion Propagation}. \cite{Nguyen2017TBD} discovers frequent patterns of congestion propagations in the traffic networks by a) discovering spatio-temporal congestions and causal relationships between them, b) revealing recurrent congestion patterns in the road network based on frequent subtree algorithm, and c) modelling the traffic congestion propagation probability. [SIGSPATIAL'17] takes a road network, the congestion conditions (when becoming congested, 20km/h) on each road segment spanning several days as input, and aims to infer the cascading patterns between road segments. 


\section{Conclusions}
\label{sec-conc}
Conclusions.


\eat{%%%%%%%%%%%%EAT
\begin{acks}
%\stitle{Acknowledgments}.
This work is supported in part by  973 program ({\small No. 2014CB340300}), NSFC ({\small No. 61322207\&61421003}),  Special Funds of Beijing Municipal Science \& Technology Commission, and MSRA Collaborative Research Program.
\end{acks}
}%%%%%%%%%%%%%%%%EAT


%
% The following two commands are all you need in the
% initial runs of your .tex file to
% produce the bibliography for the citations in your paper.
%\newpage
%\clearpage
\balance
\bibliographystyle{ACM-Reference-Format}
\bibliography{paper}
% You must have a proper ".bib" file
%  and remember to run:
% latex bibtex latex latex
% to resolve all references
%
% ACM needs 'a single self-contained file'!
%



%APPENDICES are optional
%\balancecolumns

%\input{sec-app}   %%use this one for appendix



\end{document}
