\documentclass[sigconf]{acmart}

\usepackage{booktabs} % For formal tables


%\documentclass[10pt,conference,balance,letterpaper]{sig-alternate-05-2015}
\special{papersize=8.5in,11in}
%\usepackage{latexsym}
%\usepackage{amsfonts}
%\usepackage{amsmath}
%\usepackage{amssymb}
\usepackage{color}
\usepackage{epsfig}
\usepackage{xspace}
\usepackage{graphicx,epstopdf}
\usepackage{subfigure}
%\usepackage{cite}
\usepackage{balance}
%\usepackage{amsmath, bm}
\usepackage[english]{babel}

\pdfpagewidth=8.5in
\pdfpageheight=11in

%%For removing copyright
%\usepackage{blindtext}
%\usepackage{etoolbox}
%\makeatletter
%\patchcmd{\maketitle}{\@copyrightspace}{}{}{}
%\makeatother
%%%%%End of removing copyright


% ALGORITHMS
%%%%%%%%%%%%%%%%%%%%%%%%%%%%%%%%%%%%%%%%%%%%%%%%%%%%%%%%%%%%%%%%%%%%%%%%%%%%%%%
\newcommand{\SELECT}{\mbox{{\bf select}}\ }
\newcommand{\FROM}{\mbox{{\bf from}\ }}
\newcommand{\WHERE}{\mbox{\bf where}\ }
\newcommand{\SUM}{\mbox{{\bf sum}}\ }
\newcommand{\GROUPBY}{\mbox{{\bf group by}}\ }
\newcommand{\HAVING}{\mbox{{\bf having}}\ }
\newcommand{\CASE}{\mbox{{\bf case}}\ }
\newcommand{\END}{\mbox{{\bf end}}\ }
\newcommand{\WHEN}{\mbox{{\bf when}}\ }
\newcommand{\EXISTS}{\mbox{{\bf exists}}\ }
\newcommand{\COUNT}{\mbox{\kw{count}}}
\newcommand{\INSERTINTO}{\mbox{{\bf insert into}}\ }
\newcommand{\UPDATE}{\mbox{{\bf update}}\ }
\newcommand{\SET}{\mbox{{\bf set}}\ }
\newcommand{\IN}{\mbox{{\bf in}}\ }
\newcommand{\If}{\mbox{\bf if}\ }
\newcommand{\Let}{\mbox{\bf let}\ }
\newcommand{\Call}{\mbox{\bf call}\ }
\newcommand{\Then}{\mbox{\bf then}\ }
\newcommand{\To}{\mbox{\bf to}\ }
\newcommand{\Else}{\mbox{\bf else}\ }
\newcommand{\ElseIf}{\mbox{\bf elseif}\ }

\newcommand{\While}{\mbox{\bf while}\ }
\newcommand{\Begin}{\mbox{\bf begin}\ }
\newcommand{\End}{\mbox{\bf end}\ }
\newcommand{\Do}{\mbox{\bf do}\ }
\newcommand{\Downto}{\mbox{\bf downto}\ }
\newcommand{\Repeat}{\mbox{\bf repeat}\ }
\newcommand{\Until}{\mbox{\bf until}\ }
\newcommand{\For}{\mbox{\bf for}\ }
\newcommand{\Merge}{\mbox{\bf merge}\ }
\newcommand{\Replace}{\mbox{\bf replace}\ }
\newcommand{\Remove}{\mbox{\bf remove}\ }
\newcommand{\Find}{\mbox{\bf find}\ }
\newcommand{\Each}{\mbox{\bf each}\ }

\newcommand{\ForEach}{\mbox{\bf for each}\ }
\newcommand{\Or}{\mbox{\bf or}\ }
\renewcommand{\And}{\mbox{\bf and}\ }
\newcommand{\Not}{\mbox{\bf not}\ }
\newcommand{\Break}{\mbox{\bf break}\ }
\newcommand{\Continue}{\mbox{\bf continue}\xspace}
\newcommand{\Return}{\mbox{\bf return}\ }
\newcommand{\Case}{\mbox{\bf case}\ }
\newcommand{\Of}{\mbox{\bf of}\ }
\newcommand{\EndCase}{\mbox{\bf end-case}\ }
\newcommand{\NIL}{\mbox{\em nil}}
\newcommand{\False}{\mbox{\em false}}
\newcommand{\True}{\mbox{\em true}}
\newcommand{\algAND}{{\sc and}\xspace}
\newcommand{\OR}{{\sc or}\xspace}
\newcommand{\NOT}{{\sc not}\xspace}
%%%%%%%%%%%%%%%%%%%%%%%%%%%%%%%%%%%%%%%%%%%%%%%%%%%%%%%%%%%%%%%%% ALGORITHMS

\newcommand{\spara}[1]{\smallskip\noindent{\bf #1}}
\newcommand{\eat}[1]{}
\newcommand{\squishlist}{
 \begin{list}{$\bullet$}
  {  \setlength{\itemsep}{0pt}
     \setlength{\parsep}{3pt}
     \setlength{\topsep}{3pt}
     \setlength{\partopsep}{0pt}
     \setlength{\leftmargin}{2em}
     \setlength{\labelwidth}{1.5em}
     \setlength{\labelsep}{0.5em}
} }
\newcommand{\squishlisttight}{
 \begin{list}{$\bullet$}
  { \setlength{\itemsep}{0pt}
    \setlength{\parsep}{0pt}
    \setlength{\topsep}{0pt}
    \setlength{\partopsep}{0pt}
    \setlength{\leftmargin}{2em}
    \setlength{\labelwidth}{1.5em}
    \setlength{\labelsep}{0.5em}
} }

\newcommand{\squishdesc}{
 \begin{list}{}
  {  \setlength{\itemsep}{0pt}
     \setlength{\parsep}{3pt}
     \setlength{\topsep}{3pt}
     \setlength{\partopsep}{0pt}
     \setlength{\leftmargin}{1em}
     \setlength{\labelwidth}{1.5em}
     \setlength{\labelsep}{0.5em}
} }

\newcommand{\squishend}{
  \end{list}
}
\newcommand{\sttab}{\rule{0pt}{8pt}\\[-3ex]}
\newcounter{ccc}
\newcommand{\bcc}{\setcounter{ccc}{1}\theccc.}
\newcommand{\icc}{\addtocounter{ccc}{1}\theccc.}
\newcommand{\myhrule}{\rule[.5pt]{\hsize}{.5pt}}
\newcommand{\mat}[2]{{\begin{tabbing}\hspace{#1}\=\+\kill #2\end{tabbing}}}
\newcommand{\stitle}[1]{\vspace{0.5ex}\noindent{\bf #1}}
\newcommand{\etitle}[1]{\vspace{0.5ex}\noindent{\em \underline{#1}}}
\newcommand{\marked}[1]{\textcolor{red}{#1}}
\newcommand{\markedb}[1]{\textcolor{blue}{#1}}
\newcommand{\subsubtitle}[1]{\vspace{0.5ex}\noindent\underline{{\bf #1}}}

%\newcommand{\stab}{\rule{0pt}{8pt}\\[-1.6ex]}
%\newcommand{\sttab}{\rule{0pt}{8pt}\\[-2ex]}
\newcommand{\sstab}{\rule{0pt}{8pt}\\[-2ex]}
\newcommand{\bi}{\begin{itemize}}
\newcommand{\ei}{\end{itemize}}
\renewcommand{\ni}{\noindent}


\newcommand{\nthesection}{\arabic{section}}
\newcounter{theoremc}
\renewcommand{\thetheorem}{\arabic{theoremc}}
\newcounter{prop}
\renewcommand{\theprop}{\thetheorem}
\newcounter{property}
\renewcommand{\theprop}{\thetheorem}
\newcounter{lemma}
\renewcommand{\thelemma}{\thetheorem}
\newcounter{cor}
\renewcommand{\thecor}{\thetheorem}
\renewenvironment{theorem}{\begin{em}
        \refstepcounter{theoremc}
        {\vspace{1ex} \noindent\bf  Theorem  \thetheorem:}}{
        \end{em}\eop\vspace{.5ex}} %\hspace*{\fill}\vspace*{1ex}}
\newenvironment{prop}{\begin{em}
        \refstepcounter{theoremc}
        {\vspace{1ex}\noindent \bf Proposition \theprop:}}{
        \end{em}\eop\vspace{0.5ex}}%\hspace*{\fill}\vspace*{1ex}}
\renewenvironment{lemma}{\begin{em}
        \refstepcounter{theoremc}
        {\vspace{1ex}\noindent\bf Lemma \thelemma:}}{
        \end{em}\eop\vspace{0.5ex}} %\hspace*{\fill}\vspace*{1ex}}
\newenvironment{cor}{\begin{em}
        \refstepcounter{theoremc}
        {\vspace{1ex}\noindent\bf Corollary \thecor:}}{
        \end{em}\eop\vspace{0.5ex}} %\hspace*{\fill}\vspace*{1ex}}
\newcounter{claim}
\renewcommand{\theclaim}{\thetheorem}
\newenvironment{claim}{\begin{em}
        \refstepcounter{theoremc}
        {\vspace{1ex}\noindent \bf Claim \theclaim:}}{
        \end{em}\eop\vspace{0.5ex}}%\hspace*{\fill}\vspace*{1ex}}
\newcounter{definition}[section]
\renewcommand{\thedefinition}{\nthesection.\arabic{definition}}
\renewenvironment{definition}{
        \vspace{1.5ex}
        \refstepcounter{definition}
        {\noindent\bf Definition {\bf \thedefinition}:}}{\eop\vspace{1.5ex}
}
\newcounter{alg}[section]
\renewcommand{\thealg}{\nthesection.\arabic{alg}}
\newenvironment{alg}[1]{
        \refstepcounter{alg}
        {\vspace{1ex}\noindent\bf Algorithm \thealg:\, #1}}{
        \vspace*{1ex}}
\newcounter{arule}
\renewcommand{\thearule}{\arabic{arule}}
\newenvironment{arule}{
        \vspace{0.6ex}
        \refstepcounter{arule}
        {\noindent \em Rule \thearule:}}{
        }
%\renewenvironment{proof}{
\renewenvironment{proof}{
        \vspace{.5ex}
        {\noindent\bf Proof:}}{\eop\vspace{.5ex}}
\newenvironment{proofS}{
        \vspace{0ex}
        {\noindent\bf Proof:}}{\eop\vspace{0ex}}
\newenvironment{proofSketch}{
        \vspace{.5ex}
        {\noindent\bf Proof Sketch:}}{\eop\vspace{.5ex}}
\newenvironment{property}{
        \vspace{1ex}
        {\noindent\bf Property:}}{\eop\vspace{1ex}}


\newcommand{\lemmachar}{{\unskip\nobreak\hfil\penalty50\hskip1em\hbox{}%
\nobreak\hfil\rule{1.2ex}{1.4ex}\hfil%
\parfillskip=0pt \finalhyphendemerits=0 \par}}


\newcommand{\eop}{\hspace*{\fill}\mbox{$\Box$}}     % End of proof
\newcounter{example}
\renewcommand{\theexample}{\arabic{example}}
\renewenvironment{example}{
        \vspace{1ex}
        \refstepcounter{example}
        {\noindent\bf Example \theexample:}}{
        \eop\vspace{0.5ex}}

\newcommand{\ie}{\emph{i.e.,}\xspace}
\newcommand{\eg}{\emph{e.g.,}\xspace}
\newcommand{\wrt}{\emph{w.r.t.}\xspace}
\newcommand{\aka}{\emph{a.k.a.}\xspace}
\newcommand{\kw}[1]{{\ensuremath {\mathsf{#1}}}\xspace}

%baselines
\newcommand{\ewpr}{{\sc EWPR}\xspace}
\newcommand{\ewprall}{$\kw{EWPR^*}$}
\newcommand{\blpr}{{\sc PR}\xspace}
\newcommand{\blwpr}{{\sc WPR}\xspace}
\newcommand{\blmulrank}{\kw{MulRank}}

\newcommand{\pagerank}{\kw{PRank}} %PageRank
\newcommand{\futurerank}{\kw{FRank}} %FutureRank
\newcommand{\hhgrank}{\kw{HRank}} %HHGBiRank
\newcommand{\ensemblerank}{\kw{SARank}}
\newcommand{\batensemble}{\kw{batSARank}}
\newcommand{\incensemble}{\kw{incSARank}}
\newcommand{\powensemble}{\kw{powSARank}}

\newcommand{\scc}{{\sc SCC}\xspace}
\newcommand{\sccs}{{\sc SCCs}\xspace}

\newcommand{\twprscc}{\kw{bat}{\sc TWPR}\xspace}
\newcommand{\inctwprscc}{\kw{inc}{\sc TWPR}\xspace}
\newcommand{\powtwprscc}{\kw{pow}{\sc TWPR}\xspace}
\newcommand{\PairAcc}{\kw{PairAcc}}

%dataset
\newcommand{\magdata}{{\sc MAG}\xspace}
\newcommand{\aan}{{\sc AAN}\xspace}
\newcommand{\aminer}{{\sc DBLP}\xspace}
\newcommand{\recom}{{\sc Recom}\xspace}
\newcommand{\fcita}{{\sc FCita}\xspace}




% Copyright
%\setcopyright{none}
%\setcopyright{acmcopyright}
%\setcopyright{acmlicensed}
\setcopyright{rightsretained}
%\setcopyright{usgov}
%\setcopyright{usgovmixed}
%\setcopyright{cagov}
%\setcopyright{cagovmixed}


% DOI
\acmDOI{10.475/123_4}

% ISBN
%\acmISBN{123-4567-24-567/08/06}

%Conference
\acmConference[KDD'18]{ACM KDD conference}{August, 2018}{London, United Kingdom.}
\acmYear{2018}
\copyrightyear{2018}

\acmPrice{15.00}



\begin{document}


\title{Multi-Scenario Traffic Speed and Flow Prediction}

\title{Towards Expected Fastest Route Computation}



\eat{
\author{
Renjun Hu$^{1,2}$, Junming Liu$^3$, Hui Xiong$^3$, Shuai Ma$^{1,2}$, Jinpeng Huai$^{1,2}$}
\affiliation{%
  \institution{$^1$ SKLSDE Lab, Beihang University, China}
  \institution{$^2$ Beijing Advanced Innovation Center for Big Data and Brain Computing, China}
  \institution{$^3$ Rutgers University, USA}
  \institution{\{hurenjun, mashuai, huaijp\}@buaa.edu.cn \hspace{12ex} \{jl1433, hxiong\}@rutgers.edu}
}
}%%%%%EAT



\author{Renjun Hu}
\affiliation{%
  \institution{SKLSDE Lab, Beihang University}
}
\email{hurenjun@buaa.edu.cn}
%
\author{Junming Liu}
\affiliation{%
  \institution{Rutgers University}
}
\email{jl1433@rutgers.edu}
%
\author{Hui Xiong}
\affiliation{%
  \institution{Rutgers University}
}
\email{hxiong@rutgers.edu}
%
\author{Jingjing Gu}
\affiliation{%
  \institution{Nanjing University of Aeronautics and Astronautics}
}
\email{gujj@nuaa.edu.cn}
%
\author{Shuai Ma}
\affiliation{%
  \institution{SKLSDE Lab, Beihang University}
}
\email{mashuai@buaa.edu.cn}
%
\author{Jinpeng Huai}
\affiliation{%
  \institution{SKLSDE Lab, Beihang University}
}
\email{huaijp@buaa.edu.cn}


% The default list of authors is too long for headers}
\renewcommand{\shortauthors}{R. Hu et al.}


\begin{abstract}
Abstract.
\end{abstract}

\maketitle

\input{manuscript}
\section{Introduction}
\label{sec-intro}

Modeling traffic flows is one of the most fundamental issues for urban transportation problems. It aims to continuously predict the traffic flows on each road segments. An accurate traffic flow prediction model can benefit numerous real-life tasks, such as travel time estimation, intelligent driving direction, traffic congestion detection and urban planning~\cite{Zheng2014TIST}.

Traffic flow is XXXXXXX in real-life. {\em reasons and examples of dynamism.} {\em problems of ignoring the dynamism}

{\bf Existing models} predict the flows on road segments at specific time slot. The main idea behind these models is that ``similar" road segments share similar traffic condition, and road segments at the same time slot share similar traffic. Specifically, these model exploit low rank matrix decomposition to reveal the hidden structure of traffic condition matrix which records the traffic condition of each road segment spanning a period of time. However, the derived latent factors only preserve the main structure of the matrix, and, hence, cannot deal with the traffic flows in {\em anomalous cases}, \eg road works, events, accidents.

\stitle{Goal}. To predict the city-wide traffic condition, \ie the number of vehicles and the traveling speed of each road segments.

Modeling of traffic condition:
a)	the flow speed on this link within the time slot to indicate this traffic condition;
b)	both travel speed and traffic volume;
c)	Density, length of queue, etc.

\stitle{Challenges}.
\bi
\item Traffic is xxxxxxx, \eg traffic accident, road work, administrative control, and event, holiday (volume and capacity)
\item detour behaviors
\item Data problems: Data is sparse and uneven on both time and space (topology + trajectory), Individual trajectory is noisy and unreliable (group behavior)
\ei

\stitle{Contributions}. To this end,

\ni (1) Data-driven transmission model. For each intersection, total in-flow and out-flows, and further distinguish the normal and anomalous flows, spatio-temporal correlations

\ni (2) prediction model

\ni (3)


\section{Experimental Study}
\label{sec-exp}

Using two real-life datasets, we conduct an extensive experimental study of , compared with the state of the art methods.




%\stitle{Related work}.
\section{Related work}
\label{sec-related}

We summarize related work as follows.

\stitle{Traffic Speed Prediction}. The problem of traffic prediction is of great interest in recent years due to both the availability of big traffic data and its fundamental importance for a broad range of urban computing tasks~\cite{Zheng2014TIST}.
%
Since traffic speed is essentially a type of time series data, time series analysis has been extensively exploited in this domain~\cite{traffic2011TRP,TS2012ICDM}. Notably, in~\cite{TS2012ICDM} the authors enhance the tradition auto-regression integrated moving average (ARIMA) model to further incorporate the knowledge from both historical data and traffic incidents.


Latent Space Model~\cite{LSM2016KDD}.
Muti-task Learning~\cite{MTL2017ICDM}.
Matrix decomposition~\cite{Zhu2013TMC,Shang2014KDD} and Tensor decomposition~\cite{Wang2014KDD}.
Deep Learning~\cite{DL2016ICDM,DL2017SDM,DL2017CoRR}

%\cite{traffic2011TRP}: multivariate spatial-temporal autoregressive, at a fine granularity (5 min) and over multiple time periods (12 * 5min)
%\cite{TS2012ICDM}: short-term t+1 (1 time interval, e.g., 5min), long-term t+(>1)

\stitle{Route Planning}.

\stitle{Others}. 




%\stitle{Mobility pattern}.

%\stitle{Traffic Condition Modeling}. On individual road segments. On the entire road network: interpolation-based methods consider spatial correlation among different locations' traffic condition and the compressive sensing-based approaches~\cite{Zhu2013TMC} consider both spatial and temporal correlation (by filling empty values in a road-time matrix). The work in~\cite{Shang2014KDD} further incorporates two sets of additional knowledge, \ie geographic contexts of road segments and traffic pattern revealing the correlations between different time slots.

%\stitle{Traffic Congestion Propagation}. \cite{Nguyen2017TBD} discovers frequent patterns of congestion propagations in the traffic networks by a) discovering spatio-temporal congestions and causal relationships between them, b) revealing recurrent congestion patterns in the road network based on frequent subtree algorithm, and c) modelling the traffic congestion propagation probability. [SIGSPATIAL'17] takes a road network, the congestion conditions (when becoming congested, 20km/h) on each road segment spanning several days as input, and aims to infer the cascading patterns between road segments. 


\section{Conclusions}
\label{sec-conc}
Conclusions.


\eat{%%%%%%%%%%%%EAT
\stitle{Acknowledgments}.
This work is supported in part by  973 program ({\small No. 2014CB340300}), NSFC ({\small No. 61322207\&61421003}),  Special Funds of Beijing Municipal Science \& Technology Commission, and MSRA Collaborative Research Program.
}%%%%%%%%%%%%%%%%EAT


%
% The following two commands are all you need in the
% initial runs of your .tex file to
% produce the bibliography for the citations in your paper.
%\newpage
%\clearpage
\balance
\bibliographystyle{ACM-Reference-Format}
\bibliography{paper}
% You must have a proper ".bib" file
%  and remember to run:
% latex bibtex latex latex
% to resolve all references
%
% ACM needs 'a single self-contained file'!
%



%APPENDICES are optional
%\balancecolumns

%\input{sec-app}   %%use this one for appendix



\end{document}
