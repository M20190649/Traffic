\section{Introduction}
\label{sec-intro}

{\bf Goals}.

(a) discovering patterns shared by efficient individuals;

(b) assessing/evaluating human productivity.

{\bf Data}.

(a) Twitter data: daily behaviors, \eg reading and bodybuilding, and their diversity;

(b) Linkedin: how to evaluate productivity?

(c) DBLP: peoples with more publications are regarded as more efficient;

(d) Other sources.


{\bf Dimensions of productivity}.

(a) productivity includes effectiveness (producing the right products or services), efficiency (prudent utilization of resources), and quality (meeting technical and customer specifications).
https://www.nap.edu/read/2135/chapter/6\#106

(b) in Conceptual Productivity model, productivity is a function of four major factors: task capacity (technology, task design, physical inputs), individual capacity (knowledge, skills, abilities), individual effort (attitudes, beliefs), and uncontrollable interferences.
https://www.nap.edu/read/2135/chapter/6\#112


\stitle{Contributions}.

\ni (1)

\ni (2)

\ni (3) 